
    \documentclass{article}
    \usepackage{amsmath}
    \usepackage{graphicx}

    \begin{document}

    \textbf{OpenAI Response:}

    \begin{quote}
    Aby rozwiązać to zadanie, musimy użyć zasady zachowania objętości (przy założeniu, że masa i gęstość się nie zmieniają). Mamy dwa etapy:

1. **Obliczenie objętości pręta przed rozciąganiem:**

Masa \( m \) = 10 g = 0.01 kg

Gęstość \( \rho \) = 19 300 kg/m\(^3\)

Objętość \( V \) = \(\frac{m}{\rho}\) = \(\frac{0.01 \text{ kg}}{19 300 \text{ kg/m}^3} = 5.18 \times 10^{-7} \text{ m}^3\)

2. **Obliczanie długości drutu po rozciągnięciu:**

Średnica drutu = 5 μm = \(5 \times 10^{-6}\) m

Promień drutu \( r \) = \(\frac{5 \times 10^{-6}}{2}\) m = \(2.5 \times 10^{-6}\) m

Objętość \( V \) (pozostaje taka sama) = Pole przekroju × długość

Pole przekroju \( A \) = \(\pi r^2\)

\( A = \pi (2.5 \times 10^{-6})^2 \)

\( A \approx 1.96 \times 10^{-11} \text{ m}^2\)

Długość drutu \( L \) = \(\frac{V}{A} = \frac{5.18 \times 10^{-7}}{1.96 \times 10^{-11}} \approx 26\,428 \text{ m}\)

Ostatecznie, długość drutu po rozciągnięciu wynosi około \(26\,428\) metrów.
    \end{quote}

    \end{document}
    